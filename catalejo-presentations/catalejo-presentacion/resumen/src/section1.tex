\section*{¿Quienes somos?}

Somos un equipo que crea y/o adapta experiencias de aprendizaje a poblaciones específicas a partir de metodologías de aprendizaje activo a través herramientas tecnologías que propician el entorno adecuado para el desarrollo de habilidades del siglo XXI


\section*{¿Qué hacemos?}
Somos un equipo que crea un espacio propicio para adaptar tecnologías del contexto 4.0 en procesos de aprendizaje utilizando pedagogías activas enmarcadas en los Objetivos de Desarrollo Sostenible (ODS).



\section*{¿Cómo lo hacemos?}
Brindando herramientas para solución de problemas cotidianos con base en el desarrollo de habilidades como  creatividad, liderazgo, trabajo colaborativo,  pensamiento crítico y habilidades comunicativas.

\section*{¿Qué tenemos?}
Un equipo con talento humano trabajando desde la interdisciplinariedad para ofrecer soluciones pertinentes para la comunidad.
¿Cómo se percibe?
Puede acercarse a la solución desde tres puntos:
\begin{itemize}
  \item Propuesta metodológica para talleres de formación para talleres de trabajo con las comunidades objetivo.
  \item Kits de diseño para auto-aprendizaje utilizando las herramientas virtuales que ofrecemos. 
  \item Diseño de Hardware/Software según requerimientos del cliente.
\end{itemize}

\section*{Metodología aplicada}

Nosotros pretendemos que estudiantes de educación secundaria se conviertan en solucionadores de problemas a partir de metodologías disruptivas
que ataquen tanto el analfabetismo funcional como el analfabetismo tecnológico en un contexto de cuarta revolución industrial teniendo en
cuenta los objetivos de desarrollo sostenible. Se busca que éstos últimos sean abordados por cada estudiante desde su contexto, lo que
permite que se apropien de los ODS en su cotidianidad. Y a partir de herramientas open source en las cuales nos apoyamos y/o desarrollamos,
el estudiante puede generar alternativas de solución a los problemas en su ámbito local.
Para ello, desde la parte pedagógica, utilizamos metodologías de aprendizaje basado tanto en problemas como en proyectos teniendo en cuenta una
investigación y documentación de el estado del arte, la historia y la evolución de las tecnologías. De tal manera, que la principal herramienta
transversal es el Lenguaje, pues nos permite a todos ser parte de una solución manteniendo una comunicación efectiva según el paradigma y el
receptor, sea máquina o humano.
Nuestro principal objetivo no es enseñarles a usar las herramientas, si no que ellos tengan conductas de auto-aprendizaje y motivación donde 
den inicio al descubrimiento o fortalecimiento de las habilidades que se requieren para este contexto, entre ellas: Pensamiento crítico,
trabajo colaborativo, habilidades comunicativas, creatividad  y una muy importante que es el liderazgo. Lo que se pretende es que el rol de
aquél que transfiere información y conocimiento se transforme en un facilitador, de tal manera que los estudiantes como comunidad sean
responsables de generar un conocimiento.
En el contexto mencionado hemos desarrollado unos programas metodológicos que permiten que los estudiantes puedan abordar proyectos,
se motiven y generen alternativas de solución. Estas metodologías se han implementado tanto en colegios como en clubes. Además de la
metodología, se desarrolló una herramienta (hardware/software), con el nombre de CATALEJO EDITOR con la que se busca acelerar el proceso
de estructuración del pensamiento donde el estudiante antes de tener que aprender un lenguaje de programación para poder decirle a una
máquina qué debe hacer se enfoca en entender y utilizar bloques gráficos con lenguaje natural, permitiendo que el estudiante se enfoque
en el qué quiere que haga la máquina para generar alternativas de solución tangibles a los problemas de objetivos de desarrollo sostenible.

\section*{Productos}

\subsection*{Catalejo Editor: La programación desde la música y el arte}

Se presentará una herramienta de programación gráfica y versátil que permite a quien la use desarrollar proyectos relacionados con el arte, desde hacer música (armonías, melodías) hasta hacer su propio instrumento con una placa de desarrollo; a continuación se presenta el contexto y justificación.

En  el contexto actual-global que está inmersa la sociedad se presentan nuevos retos que demandan el desarrollo de habilidades sociales. Como sociedad estamos viviendo el desarrollo de la “economía del conocimiento”; quien es capaz de usar herramientas para filtrar de la información conocimiento convierte estos retos en oportunidades que podrían dar como  consecuencia una mejor calidad de vida.

Aunque por nuestra naturaleza tenemos diferentes intereses, aprender programación (una de las herramienta de ésta era del conocimiento) permitiría posiblemente una oportunidad mayor de adaptación para enfrentar las necesidades cambiantes de la sociedad que aquel que no lo haga.

En experiencias vividas con estudiantes, aquellos que tienen un pensamiento estructurado, lógico, que muestran interés por las matemáticas son los primeros en arriesgarse en éste mundo de la programación, mientras aquellos que muestran una inclinación por gusto al Arte, habilidades manuales, deportivas entre otras, no les despierta mayor emoción.

Para romper el anterior paradigma y despertar interés por aquellos que tienen el gusto por el Arte hemos desarrollado una herramienta que integra diferentes programas los cuales permiten crear música, sintetizando diferentes instrumentos, construir instrumentos que respondan a nuevas expresiones (como aparatos con solenoides, luces para música, música abstracta), así entonces, si el estudiante desarrolla un proyecto que responda a sus intereses con ésta herramienta, implícitamente requiere aprender programación, lo cuál será el siguiente paso pues cuenta con un factor determinante como lo es la motivación.

\url{https://music-and-programation-course.readthedocs.io/es/latest/}

\subsection*{Medialab}

Kit de desarrollo educativo para el aprendizaje de temas relacionados a robótica e IOT con una interfaz de programación intuitiva,
código y hardware es opensource.

\url{https://sourceforge.net/projects/medialab-unal/} 

\subsection*{AppNode}

es un conjunto de software tanto para la tarjeta de desarrollo MediaLab como para la interfaz
de usuario que permite su programación. AppNode se desarrolla con el fin de darle la facilidad al usuario
de una programación sin preocuparse de la sintaxis a partir de bloques gráficos que representan funciones
posibles por la tarjeta como es el caso de Internet de las cosas. La principal característica de AppNode
es que se ejecuta como una aplicación al lado del servidor, que en el caso de una aula de clase puede
ser prestado por una raspberry pi de ejemplo.

\url{https://bitbucket.org/pinguinotux/appnode}

\subsection*{LuaBot}

Plataforma que puede ser usada en robótica, domótica u otros fines
donde sea necesario automatizar un proceso y pueda intervenir la electrónica.
Consta tanto de un hardware y un software, ambos de tipo open source. Es
pensado como una herramienta que facilita la programación de tareas que
necesiten ser automatizadas y/o tele-operadas incorporando elementos básicos
para tal fin, además, evita al usuario final la engorrosa tarea de aprender
un lenguaje de programación, haciendo que éste se ocupe principalmente del
algoritmo a desarrollar.

\url{https://bitbucket.org/pinguinotux/luabot} 


